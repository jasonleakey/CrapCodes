% !Mode:: "TeX:UTF-8" 

%避免宏包 hyperref 和 arydshln 不兼容带来的目录链接失效的问题。
%\def\temp{\relax}
%\let\temp\addcontentsline
%\gdef\addcontentsline{\phantomsection\temp}

\makeatletter
\gdef\hitempty{}

%重新定义BiChapter命令,可实现标题手动换行,但不影响目录
%\def\BiChapter{\relax\@ifnextchar [{\@BiChapter}{\@@BiChapter}}
%\def\@BiChapter[#1]#2#3{\chapter[#1]{#2}
%    \addcontentsline{toe}{chapter}{\bfseries \xiaosi Chapter \thechapter \hspace{0.5em} #3}}
%\def\@@BiChapter#1#2{\chapter{#1}
%    \addcontentsline{toe}{chapter}{\bfseries \xiaosi Chapter \thechapter \hspace{0.5em}{\boldmath #2}}}
%
%\newcommand{\BiSection}[2]
%{   \section{#1}
%    \addcontentsline{toe}{section}{\protect\numberline{\csname thesection\endcsname}#2}
%}

%\newcommand{\BiSubsection}[2]
%{    \subsection{#1}
%    \addcontentsline{toe}{subsection}{\protect\numberline{\csname thesubsection\endcsname}#2}
%}

%\newcommand{\BiSubsubsection}[2]
%{    \subsubsection{#1}
%    \addcontentsline{toe}{subsubsection}{\protect\numberline{\csname thesubsubsection\endcsname}#2}
%}

%\newcommand{\BiAppendixChapter}[2] % 该附录命令适用于发表文章,简历等
%{\phantomsection
%\markboth{#1}{#1}
%\addcontentsline{toc}{chapter}{\xiaosi #1}
%\addcontentsline{toe}{chapter}{\bfseries \xiaosi #2}  \chapter*{#1}
%}

%\newcommand{\BiAppChapter}[2]    % 该附录命令适用于有章节的完整附录
%{\phantomsection 
% \chapter{#1}
% \addcontentsline{toe}{chapter}{\bfseries \xiaosi Appendix \thechapter~~#2}
%}

%\renewcommand{\thefigure}{\arabic{chapter}-\arabic{figure}}%使图编号为 7-1 的格式 %\protect{~}
%\renewcommand{\thesubfigure}{\alph{subfigure})}%使子图编号为 a)的格式
%\renewcommand{\p@subfigure}{\thefigure~} %使子图引用为 7-1 a) 的格式,母图编号和子图编号之间用~加一个空格
%\renewcommand{\thetable}{\arabic{chapter}-\arabic{table}}%使表编号为 7-1 的格式
%\renewcommand{\theequation}{\arabic{chapter}-\arabic{equation}}%使公式编号为 7-1 的格式


%\newcommand{\algoenname}{Algo.} %算法英文标题
%\newfloatlist[chapter]{algoen}{aen}{\listalgoenname}{\algoenname}
%\newfixedcaption{\algoencaption}{algoen}
%\renewcommand{\thealgoen}{\thechapter-\arabic{algocf}}
%\renewcommand{\@cftmakeaentitle}{\chapter*{\listalgoenname\@mkboth{\bfseries\listalgoenname}{\bfseries\listalgoenname}}}

%\renewcommand{\algorithmcfname}{算法}
%\setlength\AlCapSkip{1.2ex}
%\SetAlgoSkip{1pt}
%\renewcommand{\algocf@captiontext}[2]{\wuhao#1\algocf@typo ~ \AlCapFnt{}#2} % text of caption
%\expandafter\ifx\csname algocf@within\endcsname\relax% if \algocf@within doesn't exist
%\renewcommand\thealgocf{\@arabic\c@algocf} % and the way it is printed
%\else%                                    else
%\renewcommand\thealgocf{\csname the\algocf@within\endcsname-\@arabic\c@algocf}
%\fi
%\renewcommand{\algocf@makecaption}[2]{%中英文双标题一定多于一行,因此去掉单行时的判断,并将\parbox中标题设置为居中
%  \addtolength{\hsize}{\algomargin}%
%  \sbox\@tempboxa{\algocf@captiontext{#1}{#2}}%
%    \hskip .5\algomargin%
%    \parbox[t]{\hsize}{\centering\algocf@captiontext{#1}{#2}}% 
%  \addtolength{\hsize}{-\algomargin}%
%}
%\newcommand{\AlgoBiCaption}[2]{%直接取出自定义的中英文标题条目加入到真正的\caption 中  
%   \caption[#1]{\protect\setlength{\baselineskip}{1.5em}#1 \protect \\ Algo. \thealgocf~~ #2} % \algoencaption{#2}   
%   \addcontentsline{aen}{algoen}{\protect\numberline{\thealgoen}{#2}}
%   }

\makeatother

%定义 学科 学位
\def \xuekeEngineering {Engineering}
\def \xuekeScience {Science}
\def \xuekeManagement {Management}
\def \xuekeArts {Arts}

\ifx \xueke \xuekeEngineering
\newcommand{\cxueke}{工学}
\newcommand{\exueke}{Engineering}
\fi

\ifx \xueke \xuekeScience
\newcommand{\cxueke}{理学}
\newcommand{\exueke}{Science}
\fi

\ifx \xueke \xuekeManagement
\newcommand{\cxueke}{管理学}
\newcommand{\exueke}{Management}
\fi

\ifx \xueke \xuekeArts
\newcommand{\cxueke}{文学}
\newcommand{\exueke}{Arts}
\fi 

