% !Mode:: "TeX:UTF-8" 

\theoremstyle{plain}
\theorembodyfont{\song\rmfamily}
\theoremheaderfont{\hei\rmfamily}
\newtheorem{definition}{\hei 定义}[chapter]
\newtheorem{example}{\hei 例}[chapter]
\newtheorem{algo}{\hei 算法}[chapter]
\newtheorem{theorem}{\hei 定理}[chapter]
\newtheorem{axiom}{\hei 公理}[chapter]
\newtheorem{proposition}{\hei 命题}[chapter]
\newtheorem{lemma}{\hei 引理}[chapter]
\newtheorem{corollary}{\hei 推论}[chapter]
\newtheorem{remark}{\hei 注解}[chapter]
\newenvironment{proof}{\noindent{\hei 证明:}}{\hfill $ \square $ \vskip 4mm}
\theoremsymbol{$\square$}

\allowdisplaybreaks[4]

%\CJKcaption{gb_452} 
%\CJKtilde
\setlength{\parindent}{2em}

\arraycolsep=1.6pt

\renewcommand\contentsname{\bfseries \sanhao CONTENTS}

%\renewcommand\chaptername{\CJKprechaptername~\thechapter~\CJKchaptername}

\setcounter{secnumdepth}{4} \setcounter{tocdepth}{2}

\titleformat{\chapter}[display]{\bfseries\center\sanhao}{\MakeUppercase{\chaptertitlename}~\thechapter}{0.5em}{}
\titlespacing{\chapter}{0pt}{-5.5mm}{8mm}
%\titleformat{\section}{\xiaosan\hei}{\thesection}{0.5em}{}
%\titlespacing{\section}{0pt}{4.5mm}{4.5mm}
%\titleformat{\subsection}{\sihao\hei}{\thesubsection}{0.5em}{}
%\titlespacing{\subsection}{0pt}{4mm}{4mm}
%\titleformat{\subsubsection}{\xiaosi\hei}{\thesubsubsection}{0.5em}{}
%\titlespacing{\subsubsection}{0pt}{0pt}{0pt}


%\titlecontents{chapter}[3.8em]{}{\thecontentslabel{2.3em}}{\hspace*{-2.3em}}{\titlerule*[1pc]{.}\contentspage}
%\dottedcontents{section}[3.8em]{}{2.3em}{1pc}
%\dottedcontents{subsection}[6.1em]{}{3.2em}{1pc}
% 修改点号
%\renewcommand{\cftdot}{\ensuremath\ast}
\setlength{\cftbeforechapskip}{0em}
\setlength{\cftbeforesecskip}{0em}
\setlength{\cftbeforesubsecskip}{0em}
\renewcommand{\cftdotsep}{1.5}
\renewcommand{\cftchapdotsep}{\cftdotsep}


% 按工大标准, 缩小目录中各级标题之间的缩进,使它们相隔一个字符距离,也就是12pt
\makeatletter
%\renewcommand*\l@chapter{\@dottedtoclinebold{0}{0em}{1.8em}}%控制英文目录: 细点\@dottedtocline  粗点\@dottedtoclinebold
%\renewcommand*\l@section{\@dottedtocline{1}{1.5em}{1.8em}}
%\renewcommand*\l@subsection{\@dottedtocline{2}{2.5em}{2.5em}}



\renewcommand{\frontmatter}{
  \cleardoublepage
  %\@frontmattertrue
  \pagenumbering{Roman}}

% 只有文章主体才有页眉
\let \LaTeXmainmatter\mainmatter
\renewcommand{\mainmatter}{\LaTeXmainmatter
% 定义页眉和页脚
\newcommand{\makeheadrule}{
\rule[7pt]{\textwidth}{0.75pt} \\[-23pt]
\rule{\textwidth}{2.25pt}}
\renewcommand{\headrule}{
    {\if@fancyplain\let\headrulewidth\plainheadrulewidth\fi
     \makeheadrule}}
% 默认使用fancy风格
\pagestyle{fancy}
\addtolength{\headheight}{\baselineskip}

% 使得每个章节也有页眉, 但摘要, 致谢等章节没有页眉的功能后面实现
\fancypagestyle{plain} {
    \pagestyle{fancy}
}

\lhead{}
\chead{}
\rhead{\wuhao \leftmark}
\lfoot{}
\cfoot{\xiaosi \thepage}
\rfoot{}
}



% 设置行距和段落间垂直距离
%\renewcommand{\CJKglue}{\hskip 0.96pt plus 0.08\baselineskip} %加大字间距,使每行33个字

% 调整罗列环境的布局
%\setitemize{leftmargin=3em,itemsep=0em,partopsep=0em,parsep=0em,topsep=-0em}
%\setenumerate{leftmargin=3em,itemsep=0em,partopsep=0em,parsep=0em,topsep=0em}

% 定制浮动图形和表格标题样式
%\captionnamefont{\wuhao}
%\captiontitlefont{\wuhao}
%\captiondelim{~~}
%\captionstyle{\centering}
%\renewcommand{\subcapsize}{\wuhao}
%\setlength{\abovecaptionskip}{0pt}
%\setlength{\belowcaptionskip}{0pt}

% 自定义项目列表标签及格式 \begin{publist} 列表项 \end{publist}
\newcounter{pubctr} %自定义新计数器
\newenvironment{publist}{%%%%%定义新环境
\begin{list}{[\arabic{pubctr}]} %%标签格式
    {
     \usecounter{pubctr}
     \setlength{\leftmargin}{2.5em}     % 左边界 \leftmargin =\itemindent + \labelwidth + \labelsep
     \setlength{\itemindent}{0em}     % 标号缩进量
     \setlength{\labelsep}{1em}       % 标号和列表项之间的距离,默认0.5em
     \setlength{\rightmargin}{0em}    % 右边界
     \setlength{\topsep}{0ex}         % 列表到上下文的垂直距离
     \setlength{\parsep}{0ex}         % 段落间距
     \setlength{\itemsep}{0ex}        % 标签间距
     \setlength{\listparindent}{0pt} % 段落缩进量
    }}
{\end{list}}%%%%%

% 默认字体
\renewcommand\normalsize{
  \@setfontsize\normalsize{12pt}{12pt}
  \setlength\abovedisplayskip{4pt}
  \setlength\abovedisplayshortskip{4pt}
  \setlength\belowdisplayskip{\abovedisplayskip}
  \setlength\belowdisplayshortskip{\abovedisplayshortskip}
  \let\@listi\@listI}
\def\defaultfont{\renewcommand{\baselinestretch}{1.62}\normalsize\selectfont}
\predisplaypenalty=0  %公式之前可以换页,公式出现在页面顶部

% 封面、摘要、版权、致谢格式定义
\def\ctitle#1{\def\@ctitle{#1}}\def\@ctitle{}
\def\cdegree#1{\def\@cdegree{#1}}\def\@cdegree{}
\def\caffil#1{\def\@caffil{#1}}\def\@caffil{}
\def\csubject#1{\def\@csubject{#1}}\def\@csubject{}
\def\cauthor#1{\def\@cauthor{#1}}\def\@cauthor{}
\def\csupervisor#1{\def\@csupervisor{#1}}\def\@csupervisor{}
\def\cassosupervisor#1{\def\@cassosupervisor{{\hei 副 \hfill 导 \hfill 师} & #1\\}}\def\@cassosupervisor{}
\def\ccosupervisor#1{\def\@ccosupervisor{{\hei 联 \hfill 合\hfill 导 \hfill 师} & #1\\}}\def\@ccosupervisor{}
\def\cdate#1{\def\@cdate{#1}}\def\@cdate{}
\long\def\cabstract#1{\long\def\@cabstract{#1}}\long\def\@cabstract{}
\def\ckeywords#1{\def\@ckeywords{#1}}\def\@ckeywords{}

\def\etitle#1{\def\@etitle{#1}}\def\@etitle{}
\def\edegree#1{\def\@edegree{#1}}\def\@edegree{}
\def\eaffil#1{\def\@eaffil{#1}}\def\@eaffil{}
\def\esubject#1{\def\@esubject{#1}}\def\@esubject{}
\def\eauthor#1{\def\@eauthor{#1}}\def\@eauthor{}
\def\esupervisor#1{\def\@esupervisor{#1}}\def\@esupervisor{}
\def\eassosupervisor#1{\def\@eassosupervisor{\textbf{Associate Supervisor:} & #1\\}}\def\@eassosupervisor{}
\def\ecosupervisor#1{\def\@ecosupervisor{\textbf{Co Supervisor:} & #1\\}}\def\@ecosupervisor{}
\def\edate#1{\def\@edate{#1}}\def\@edate{}
\long\def\eabstract#1{\long\def\@eabstract{#1}}\long\def\@eabstract{}
\long\def\NotationList#1{\long\def\@NotationList{#1}}\long\def\@NotationList{}
\def\ekeywords#1{\def\@ekeywords{#1}}\def\@ekeywords{}
\def\dedicationwords#1{\def\@dedicationwords{#1}}\def\@dedicationwords{}
\def\acknowledgewords#1{\def\@acknowledgewords{#1}}\def\@acknowledgewords{}
\def\natclassifiedindex#1{\def\@natclassifiedindex{#1}}\def\@natclassifiedindex{}
\def\internatclassifiedindex#1{\def\@internatclassifiedindex{#1}}\def\@internatclassifiedindex{}
\def\statesecrets#1{\def\@statesecrets{#1}}\def\@statesecrets{}
\def\nomenclature#1{\def\@nomenclature{#1}}\def\@nomenclature{}

% 定义封面
\def\makecover{
    \begin{titlepage}
    % 封面一
   \vspace*{0.8cm}
   \begin{center}
    \centerline{\xiaoer\song \@cdegree 学位论文}

    \vspace{1cm}

    \parbox[t][2.8cm][t]{\textwidth}{
    \begin{center}\erhao\hei \@ctitle\end{center} }

    \parbox[t][3.1cm][t]{\textwidth}{ %英文标题太长时可以采用\xiaoer
    \begin{center}
        \begin{spacing}{1.3}
            \xiaoer {\sf \@etitle}
        \end{spacing}
    \end{center} }

    \parbox[t][9.4cm][t]{\textwidth}{
    \begin{center}{\xiaoer\song \@cauthor}\end{center}}

    \parbox[t][1.4cm][t]{\textwidth}{
    \begin{center}{\kai \xiaoer 哈尔滨工业大学}\end{center} }
    
    {\song \xiaoer \@cdate}

    \end{center}

    %内封
    \newpage
    \thispagestyle{empty}


    % 不缩进
    \noindent {\song \xiaosi
    \begin{tabular}{@{}r@{:}l@{}}
    国内图书分类号 & \@natclassifiedindex\\
    国际图书分类号 & \@internatclassifiedindex
    \end{tabular}}\hfill
    % 删除学校代码和密级, 110907
    %{\song \xiaosi
    %\begin{tabular}{@{}r@{:}l@{}}
    %学校代码 & 10213\\
    %密级 &  公开
    %\end{tabular}}

\begin{center}

    % 更改高度, 3.2->1.2
    \parbox[t][1.2cm][t]{\textwidth}{\begin{center} \end{center} }

    \parbox[t][1.4cm][t]{\textwidth}{\xiaoer
    \begin{center} {\song \@cdegree 学位论文 }\end{center} }

    \parbox[t][6cm][t]{\textwidth}{\erhao
    \begin{center} {\hei  \@ctitle}\end{center} }
	\parbox[t][9.8cm][b]{\textwidth}
     {\sihao
    \begin{center} \renewcommand{\arraystretch}{1.62} \song 
    \begin{tabular}{l@{:~~}l}
    {\hei \xueweishort \hfill 士\hfill 研\hfill 究\hfill 生}           & \@cauthor\\
    {\hei 导\hfill 师}                       & \@csupervisor\\
	\@cassosupervisor
	\@ccosupervisor
    {\hei 申\hfill 请\hfill 学\hfill 位} & \@cdegree\\
    {\hei 学\hfill 科\hfill、\hfill 专\hfill 业}           & \@csubject\\
    {\hei 所\hfill 在\hfill 单\hfill 位} & \@caffil\\
    {\hei 答\hfill 辩\hfill 日\hfill 期} & \@cdate\\
    {\hei 授予学位单位}                     & 哈尔滨工业大学
    \end{tabular} \renewcommand{\arraystretch}{1}
    \end{center} }
\end{center}

    % 英文封面
    \newpage
    \thispagestyle{empty}

    {
    \xiaosi\noindent Classified Index: \@natclassifiedindex \\
                  U.D.C:  \@internatclassifiedindex }
    \begin{center}
    \parbox[t][1.6cm][t]{\textwidth}{\begin{center} \end{center} }
    \parbox[t][2.4cm][t]{\textwidth}{\xiaoer
    \begin{center} {  Dissertation for the {\exueweier} Degree of Engineering}\end{center} } %与中文保持一致,删除in {\exueke}

    \parbox[t][7cm][t]{\textwidth}{\erhao
    \begin{center}
        \begin{spacing}{2.7}
            {\erhao \@etitle}
        \end{spacing}
    \end{center}}

%★★★★若信息内容不太长,不会引起信息内容分行时,使用tabular环境,否则使用下面的tabularx环境。
    {\sihao\renewcommand{\arraystretch}{1.3}
    \begin{tabular}{@{}l@{\qquad}l@{}}
    \textbf{Candidate:}                     &  \@eauthor\\
    \textbf{Supervisor:}                    &  \@esupervisor\\
	\@eassosupervisor
	\@ecosupervisor
    \textbf{Academic Degree Applied for:}   &  \@edegree\\
    \textbf{Specialty:}                     &  \@esubject\\
    \textbf{Affiliation:}                   &  \@eaffil\\
    \textbf{Date of Defence:}               &  \@edate\\
    \textbf{Degree-Conferring-Institution:} &  Harbin Institute of Technology
    \end{tabular}\renewcommand{\arraystretch}{1}}

    %{\sihao\renewcommand{\arraystretch}{1.3}
    %\begin{tabularx}{\textwidth}{@{}l@{~}X@{}}
    %\textbf{Candidate:}                     &  \@eauthor\\
    %\textbf{Supervisor:}                    &  \@esupervisor\\
    %\textbf{Academic Degree Applied for:}   &  \@edegree\\
    %\textbf{Specialty:}                     &  \@esubject\\
    %\textbf{Affiliation:}                   &  \@eaffil\\
    %\textbf{Date of Defence:}               &  \@edate\\
    %\textbf{Degree-Conferring-Institution:} &  Harbin Institute of Technology
    %\end{tabularx}\renewcommand{\arraystretch}{1}}

    \end{center}
    \end{titlepage}

%%%%%%增加一空白页
%\ifxueweidoctor
%    \newpage
%    ~~~\vspace{1em}
%    \thispagestyle{empty}
%  \fi
  
%%%%%%%%%%%%%%%%%%%   Abstract and keywords  %%%%%%%%%%%%%%%%%%%%%%%
\clearpage

%\BiAppendixChapter{摘\quad 要}{Abstract (In Chinese)}
\phantomsection
\addcontentsline{toc}{chapter}{Abstract (In Chinese)}
\chapter*{\hei \xiaoer 摘 \quad 要}

\setcounter{page}{1}
\song\defaultfont
\@cabstract
\vspace{\baselineskip}

\hangafter=1\hangindent=52.3pt\noindent
{\hei 关键词:} \@ckeywords

%%%%%%%%%%%%%%%%%%%   English Abstract  %%%%%%%%%%%%%%%%%%%%%%%%%%%%%%
\clearpage

\phantomsection
%\markboth{Abstract}{Abstract}
\addcontentsline{toc}{chapter}{Abstract}
%\addcontentsline{toe}{chapter}{\bfseries \xiaosi Abstract (In English)}  
\chapter*{\textbf{ABSTRACT}}
\@eabstract
\vspace{\baselineskip}

\hangafter=1\hangindent=60pt\noindent
{\textbf{Keywords:}}  \@ekeywords

%%%%%%%%%%%%%%%%%%%%%%% DEDICATION %%%%%%%%%%%%%%%%%%%%%%%
\clearpage
\phantomsection
\addcontentsline{toc}{chapter}{Dedication}
\chapter*{\bfseries \sanhao DEDICATION}
\@dedicationwords
\vspace{\baselineskip}

%%%%%%%%%%%%%%%%%%%%%%% ACKNOWLEDGEMENT %%%%%%%%%%%%%%%%%%%%%%%
\clearpage
\phantomsection
\addcontentsline{toc}{chapter}{Acknowledgement}
\chapter*{\bfseries \sanhao ACKNOWLEDGEMENT}
\@acknowledgewords
\vspace{\baselineskip}

%%%%%%%%%%%%%%%%%%%%%%% TABLE OF CONTENTS %%%%%%%%%%%%%%%%%%%%%
\defaultfont
%\clearpage{\pagestyle{empty}\cleardoublepage}
\pdfbookmark[0]{CONTENTS}{}
\tableofcontents    
%\clearpage{\pagestyle{empty}\cleardoublepage}

%%%%%%%%%%%%%%%%%%%%%%%% LIST OF TABLES   %%%%%%%%%%%%%%%%%%%%%%%%
\phantomsection
\addcontentsline{toc}{chapter}{List of Tables}
\listoftables

%%%%%%%%%%%%%%%%%%%%%%%% LIST OF FIGURES %%%%%%%%%%%%%%%%%%%%%%%%%%
\phantomsection
\addcontentsline{toc}{chapter}{List of Figures}
\listoffigures

} % END of makecover

%%%%%%%%%%%%%%%%%%%%%%%%%%%%%%%%%%%%%%%%%%%%%%%%%%%%%%%%%%%%%%%
% 英文目录格式
%\def\@dotsep{0.75}           % 定义英文目录的点间距
%\setlength\leftmargini {0pt}
%\setlength\leftmarginii {0pt}
%\setlength\leftmarginiii {0pt}
%\setlength\leftmarginiv {0pt}
%\setlength\leftmarginv {0pt}
%\setlength\leftmarginvi {0pt}

%\def\engcontentsname{\bfseries Contents}
%\newcommand\tableofengcontents{
%   \pdfbookmark[0]{Contents}{econtent}
%     \@restonecolfalse
%   \chapter*{\engcontentsname  %chapter*上移一行,避免在toc中出现。
%       \@mkboth{%
%          \engcontentsname}{\engcontentsname}}
%   \@starttoc{toe}%
%   \if@restonecol\twocolumn\fi
%   }

\urlstyle{same}  %论文中引用的网址的字体默认与正文中字体不一致,这里修正为一致的。

\renewcommand\endtable{\vspace{-4mm}\end@float}

%引用使用上标.
%\def \@cite#1#2{\textsuperscript{[{#1\if@tempswa , #2\fi}]}}
%\renewcommand\@citess[1]{\textsuperscript{[#1]}}

% TODO
\newcommand{\todo}[1]{\hl{\sf TODO{\mbox{#1}}}}
% 更改参考文献名称
\renewcommand{\bibname}{Reference}
% 高亮绿色字体表示需要修改
\sethlcolor{green}


\makeatother

