% !Mode:: "TeX:UTF-8" 

\usepackage{graphicx}
\usepackage[a4paper,text={150true mm,224true mm},top=35.5true mm,left=30true mm,head=5true mm,headsep=2.5true mm,foot=8.5true mm]{geometry}
\usepackage{titlesec}               % 控制标题的宏包
%\usepackage{titletoc}                   % 控制目录的宏包
\usepackage[titles,subfigure]{tocloft}                   % 控制目录的宏包
\usepackage{fancyhdr}                   % fancyhdr宏包 页眉和页脚的相关定义
%\usepackage[UTF8]{ctex}
\usepackage{color}          % 支持彩色
\usepackage{amsmath}        % AMSLaTeX宏包 用来排出更加漂亮的公式
\usepackage{amssymb}
\usepackage[below]{placeins}%允许上一个section的浮动图形出现在下一个section的开始部分,还提供\FloatBarrier命令,使所有未处理的浮动图形立即被处理
\usepackage{flafter}       % 使得所有浮动体不能被放置在其浮动环境之前,以免浮动体在引述它的文本之前出现.
\usepackage{multirow}       %使用Multirow宏包,使得表格可以合并多个row格
\usepackage{booktabs}       % 表格,横的粗线;\specialrule{1pt}{0pt}{0pt}
\usepackage{longtable}      %支持跨页的表格。
\usepackage{tabularx}
\usepackage{subfig}
%\usepackage{subfigure}%支持子图 %centerlast 设置最后一行是否居中
%\usepackage[subfigure]{ccaption} %支持双语标题
\usepackage[sort&compress,numbers]{natbib}% 支持引用缩写的宏包
%\usepackage{enumitem}       %使用enumitem宏包,改变列表项的格式
\usepackage{calc}           %长度可以用+ - * / 进行计算
\usepackage{txfonts}
\usepackage{bm}              % 处理数学公式中的黑斜体的宏包
\usepackage[amsmath,thmmarks,hyperref]{ntheorem}% 定理类环境宏包,其中 amsmath 选项用来兼容 AMS LaTeX 的宏包

%\def\atempxetex{xelatex}\ifx\atempxetex\usewhat %\def\atempxetex{xelatex} main.tex中已定义;
\usepackage[xetex,
            bookmarksnumbered=true,
            bookmarksopen=true,
            colorlinks=false,
            pdfborder={0 0 1},
            citecolor=blue,
            linkcolor=red,
            anchorcolor=green,
            urlcolor=blue,
            breaklinks=true,
            naturalnames  %与algorithm2e宏包协调
            ]{hyperref}

\usepackage[boxed,linesnumbered,algochapter]{algorithm2e}  % 算法的宏包,注意宏包兼容性,先后顺序为float、hyperref、algorithm(2e),否则无法生成算法列表 
\usepackage{algorithmic}
\usepackage{zhfontcfg}
\usepackage{setspace} % 方便调整行距的宏包
\usepackage{tabularx}
\usepackage{soul} % 高亮字体用
\usepackage{enumerate} % 使用特殊标号
