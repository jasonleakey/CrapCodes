\chapter[Pulse pattern classification]{\uppercase{Pulse Pattern Classification}}
\label{chap:five}

The pulse pattern classification is the key to achieving automated
pulse diagnosis. The selection of classifier greatly influences the
accuracy of result. Each classifier has their own applicable
conditions, so the selection should depend on the characteristics of
sample sets, e.g. the size of data set, the distribution of
samples. Hitherto the common used classifiers includes Bayesian
classifier, linear classifier, artificial neural network (ANN), k-nearest
neighbor classifier (KNN), supporting vector machine (SVM) etc. \todo{选分类器}

\section{Support vector machine classifier}

\subsection{Principle of SVM}
Support vector machine is a supervised learning method to analyze
data and classify patterns. Based on statistical theory, the standard SVM 
is a non-probabilistic binary linear classifier since it takes a set
of input data and predicts, for each given input, which of two
possible classes comprises the input.
an SVM training algorithm builds a model
that assigns new examples into one category or the other. An SVM model
is a representation of the examples as points in space, mapped so that
the examples of the separate categories are divided by a clear gap
that is as wide as possible. New examples are then mapped into that
same space and decided the category based on the side they fall on.

\subsection{Experimental design}
\subsection{SVM classification result}

TODO

\section{Linear discriminant analysis classifier}

\subsection{Principle of LDA}
\subsection{Experimental design}
\subsection{LDA classification result}

TODO

\section{K-nearest neighbor classifier}

\subsection{Principle of KNN}
\subsection{Experimental design}
\subsection{KNN classification result}
TODO

\section{}<++>

\section{Summary}

TODO
